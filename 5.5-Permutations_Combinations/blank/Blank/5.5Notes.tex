\documentclass[dvipsnames,11pt]{article}
\usepackage[utf8]{inputenc}	% Para caracteres en español
\usepackage{amsmath,amsthm,amsfonts,amssymb,amscd}
\usepackage{tikz}
\usepackage{multirow,booktabs}
%\usepackage[table]{xcolor}
\usepackage{fullpage}
\usepackage{lastpage}
\usepackage{enumitem}
\usepackage{fancyhdr}
\usepackage{mathrsfs}
\usepackage{wrapfig}
\usepackage{setspace}
\usepackage{calc}
\usepackage{multicol}
\usepackage{cancel}
\usepackage[retainorgcmds]{IEEEtrantools}
\usepackage[margin=3cm]{geometry}
\usepackage{amsmath}
\newlength{\tabcont}
\setlength{\parindent}{0.0in}
\setlength{\parskip}{0.05in}
\usepackage{empheq}
\usepackage{framed}
\usepackage[most]{tcolorbox}
\usepackage{xcolor}
\usetikzlibrary{positioning}
\usepackage{enumitem}


\colorlet{shadecolor}{orange!15}
\parindent 0in
\parskip 12pt
\geometry{margin=1in, headsep=0.25in}
\theoremstyle{definition}
\newtheorem{defn}{Definition}
\newtheorem{reg}{Rule}
\newtheorem{exer}{Exercise}
\newtheorem{note}{Note}

\newcounter{example}[section]
\newcounter{exercise}[section]

%\newcommand\hl[1]{\colorbox{pink}{#1}}
\newcommand\hl[1]{\fcolorbox{black}{white}{\color{white}#1}}


\newcommand\rmk{\colorbox{BlueGreen!15}{Note:}}

%\newcommand\ex[2]{\refstepcounter{example}
%\begin{center}
%\fcolorbox{Bittersweet}{SkyBlue}{\parbox{\textwidth}{
%\fcolorbox{Bittersweet}{White}{
%\parbox{6.35in}{\textbf{Example~\theexample: }\\#1}
%	}\vspace{5pt}
%\fcolorbox{Purple}{White}{
%\parbox{6.35in}{\textbf{Solution: }\\#2}
%	}
%	}
%	}
%\end{center}
%}


\newcommand\ex[2]{\refstepcounter{example}
\begin{center}
\fcolorbox{Bittersweet}{SkyBlue}{\parbox{\textwidth}{
\fcolorbox{Bittersweet}{White}{
\parbox{6.35in}{\textbf{Example~\theexample: }\\#1}
	}\vspace{5pt}
\fcolorbox{Purple}{White}{
\parbox{6.35in}{\textbf{Solution: }\\
	\color{white}	#2
	\vspace{
	.5in}}
	}
	}
	}
\end{center}
}

\newcommand\formula[2]{\begin{shaded}
\textbf{#1} \newline
#2
\end{shaded}
}

\newcommand\exercise[2]{\refstepcounter{exercise}
\begin{center}
\fcolorbox{DarkOrchid}{Dandelion}{\parbox{\textwidth}{
\fcolorbox{DarkOrchid}{White}{
\parbox{6.35in}{\textbf{Exercise~\theexercise: }\\#1}
	}\vspace{5pt}
\fcolorbox{Aquamarine}{White}{
\parbox{6.35in}{\textbf{Answer: }\\#2}
	}
	}
	}
\end{center}
}


\begin{document}
%\Large
\setcounter{section}{8}
\title{Section 10.2 Annuities}

\thispagestyle{empty}

\begin{center}
{\LARGE \bf Section 5.5 Permutations and combinations}\\
{\large Math 1300}\\
Fall 2019
\end{center}

\ex{\begin{enumerate}
\item Consider the letters in the set $\{a,b,c\}.$ How many different \textit{strings} of two \textit{distinct} letters can be formed? 

\item A construction crew has three members with names
$A$, $B$, and $C.$ How many \textit{different} two-person teams can be
formed from this crew?
\end{enumerate}

}
{
\begin{enumerate}
\item Since these are strings of letters, order matters. For example $ab$ and $ba$ are considered different. We have
$$ab,ac,ba,bc,ca,cb.$$
So 6.

\item In this case order does not matter: if $A$ is in a team with $B,$ then $B$ is in a team with $A.$ We only have
$$ AB, AC, BC .$$ 
So 3.
\end{enumerate}
}

In this example, we have to consider two different kinds of situations: the case when order matters and the case when order does not matter.

\section*{Permutations}

A \hl{permutation} of $n$ objects taken $r$ at a time is an arrangement of $r$ of the $n$ objects in a specific order. We will denote the number of permutations by $P(n,r).$

\ex{
Consider a baseball team of 9 players forming lines of 3,6, and 9 players. How many different ways can each of these types of lines be formed? 
}{
In this case, order matters.\\
\textbf{3 Players: } 
\begin{center}
  \begin{tabular}{ c | c }
    \hline
    Place in line 		& No. of possibilities  \\ \hline
    1					& 9 					\\ 	\hline
    2 					& 8						\\ 	\hline
    3 					& 7						\\	\hline
  \end{tabular}
\end{center}
So there are $9\cdot 8\cdot 7 = 504$ permutations.\\
Notice that this is $9\cdot (9-1) \cdot (9-2).$ 

\textbf{6 Players: }
\begin{center}
  \begin{tabular}{ c | c }
    \hline
    Place in line 		& No. of possibilities  \\ \hline
    1					& 9 					\\ 	\hline
    2 					& 8						\\ 	\hline
    3 					& 7						\\	\hline
    4 					& 6						\\	\hline
    5 					& 5						\\	\hline
    6 					& 4						\\	\hline
  \end{tabular}
\end{center} 
So there are $9\cdot 8\cdot 7\cdot 6\cdot 5\cdot 4 = 60,480$ permutations. \\
Notice that this is $9\cdot (9-1) \cdot (9-2)\cdot (9-3)\cdot(9-4)\cdot(9-5).$ 

\textbf{9 Players: }
\begin{center}
  \begin{tabular}{ c | c }
    \hline
    Place in line 		& No. of possibilities  \\ \hline
    1					& 9 					\\ 	\hline
    2 					& 8						\\ 	\hline
    3 					& 7						\\	\hline
    4 					& 6						\\	\hline
    5 					& 5						\\	\hline
    6 					& 4						\\	\hline
    7 					& 3						\\	\hline
    8 					& 2						\\	\hline
    9 					& 1						\\	\hline
  \end{tabular}
\end{center} 
So there are $9! = 362,880$ permutations. 
}

You may notice that there's a pattern to each of the solutions above.
\newpage

\formula{Permutation formula 
}
{
The number of permutations of $n$ objects taken $r$ at a time, $P(n,r),$ is given by
$$  P(n,r) = n(n-1)(n-2)\cdots (n-r+1)=\dfrac{n!}{(n-r)!}. $$
}


\ex{Compute the following:
\begin{enumerate}
\item $P(100,2)$
\item $P(6,4)$
\item $P(5,5)$
\end{enumerate}
}{
\begin{enumerate}
\item $P(100,2)=100\cdot 99 = 9900$
\item $P(6,4) = 6\cdot 5\cdot 4\cdot 3 = 360$
\item $P(5,5)=5\cdot 4\cdot 3\cdot 2\cdot 1 = 5!=120$
\end{enumerate}
}

Putting this product into the calculator is very tedious! So of course there is a way to plug it in faster.

\formula{Permutations on a calculator
}
{
You can find the number of permutations, $P(n,r),$ using a calculator by the sequence of keys
$$ \fbox{n}\; \fbox{nPr}\; \fbox{r}\;\fbox{=}. $$ 
} 

Then for $P(9,4),$ you would press
$$ \fbox{9}\; \fbox{nPr}\; \fbox{4}\;\fbox{=} $$
to get 3024.

\rmk When asked how many ways you can order $n$ objects, the answer is 
$$ P(n,n)=n! $$

\section*{Combinations}

A \hl{combination} of $n$ objects taken $r$ at a time is a selection of $r$ objects from among the $n,$ with order disregarded. The number of combinations is given by $C(n,r).$ The formula is given by

\formula{
Combination formula
}{
The number of combinations of $n$ objects taken $r$ at a time is 
$$ C(n,r) = \dfrac{P(n,r)}{r!} = \dfrac{n(n-1)(n-2)\cdots (n-r+1)}{r!}  $$
or 
$$ C(n,r) = \dfrac{n!}{r!(n-r)!} .$$
}



\ex{Compute the following:
\begin{enumerate}
\item $C(100,2)$
\item $C(6,4)$
\item $C(5,5)$
\end{enumerate}
}{
\begin{enumerate}
\item $C(100,2)= \dfrac{100\cdot 99}{2!} = 4950$
\item $C(6,4) = \dfrac{6\cdot 5\cdot 4\cdot 3}{4!} = 15 $
\item $C(5,5)=\dfrac{5\cdot 4\cdot 3\cdot 2\cdot 1}{5!}=\dfrac{5!}{5!} =1$
\end{enumerate}
}

Yet again there is a way to calculate this in one move:

\formula{Combinations on a calculator
}
{
You can find the number of permutations, $C(n,r),$ using a calculator by the sequence of keys
$$ \fbox{n}\; \fbox{nCr}\; \fbox{r}\;\fbox{=}. $$ 
} 

\section*{Applying the permutation and combination formulas}

\ex{
A high school student
decided to apply to four of the eight Ivy League colleges. In
how many possible ways can the four colleges be selected?
}
{
It does not matter what order the student applies to each college as long as they apply on time. So we have
$$ C(8,4) = 70 $$
}

\ex{
A board of directors has 10 members.
\begin{enumerate}
\item In how
many ways can a committee of 3 be chosen? 
\item In how many ways
can a chairperson, vice chairperson, and secretary be chosen?
\end{enumerate}
}{
\begin{enumerate}
\item Since the committee members can be chosen without regard to order, we have
$$ C(10,3)=120 $$
\item In this case, order matters- whoever is chosen for chairperson cannot become vice chairperson or secretary and so on. Then 
$$ P(10,3) = 720. $$ 
\end{enumerate}
}

\ex{
If 8 horses are entered in a horse race how many
different 1st, 2nd, 3rd place finishes are possible?
}{
Order matters:
$$ P(8,3) = 336 $$
}

\ex{
A political pollster wishes to draw a sample of
1500 individuals from among a population of 5,000,000
individuals.
}
{
Order does not matter when taking polls, so we have 
$$ C(5000000,1500). $$
Your calculator cannot find this answer. In fact, the number has 5934 digits.
}

\ex{
Three couples go to a movie together. In how
many ways can they be seated in 6 seats so that each couple
is seated together?
}{
Assuming the couples walk into the movie theater together, where the first couple sits will affect how the next couple needs to be arranged. Thus order matters. \\
So in how many ways can the 3 couples enter:
$$ P(3,3) = 3!. $$
But now the arrangement of each individual in each couple can be different. Each couple can sit in 2 ways. Thus by the generalized multiplication principle, we have 
$$ 3! \cdot 2\cdot 2 \cdot 2 = 48. $$ 
}

\ex{
At a benefit concert, 6
bands have volunteered to perform but there is only enough
time for four of the bands to play. How many lineups are
possible?
}
{
Since there is a lineup order matters thus
$$ P(6,4) =360. $$
}

\end{document}